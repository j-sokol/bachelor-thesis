\abstractCS{Tato bakalářská práce se zabývá průzkumem a implementací systému na~instalaci serverů bez~lidského dozoru. Cílem práce je využití některého již použitého systému a s~úpravami nasazení do produkce. V~teoterické části práce je popsána analýza 5 vybraných nástrojů. Mezi kritéria výběru těchto nástrojů patřilo z~ekonomických a praktických důvodů open-source řešení, dále rozvinutá komunita, jež může pomoci s~řešením případných problémů s~nástrojem. Na základě analýzy z~předchozí části  byl vybrán nástroj, který nejlépe vyhovoval stanoveným požadavkům: jednoduše použitelné grafické rozhraní, podpora instalace široké škály operačních systémů (podmínkou alespoň OS Debian a CentOS) a možnost rychlých úprav v~konfiguracích instalovaných serverů. Následovně je popsáno nasazení vyhovujícího nástroje Foreman, jak hlavního uzlu, tak proxy serverů v~oddělených lokalitách. V~další části je popsán vývoj a nasazení pluginu v~prostředí Foreman, který v~jeho grafickém rozhraní zobrazuje grafy z~dat sesbírané démonem collectd. Práce také obsahuje konfigurační skripty pro Ansible pro jednoduché zprovoznění serverů.}
