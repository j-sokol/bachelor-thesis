
\section{Master server}

Master server je instalován na operačním systému Debian, na serveru virtualizovaném pomocí infrastruktury KVM. Hlavní službou je nástroj Foreman, jehož instalace je dále popsána.

\subsection{Foreman}

Před samotnou instalací je třeba přidat repozitáře pro balíčkovací utilitu apt-get, jelikož verze chtěných aplikací nemusí být v~oficiálních repozitářích aktuální.

\begin{minted}{bash}
echo "deb http://deb.theforeman.org/ jessie 1.14"
    > /etc/apt/sources.list.d/foreman.list
echo "deb http://deb.theforeman.org/ plugins 1.14"
    >> /etc/apt/sources.list.d/foreman.list
apt-get -y install ca-certificates
wget -q https://deb.theforeman.org/pubkey.gpg -O- | apt-key add -
\end{minted}

Poté co jsou přidány repozitáře je možné nainstalovat  \mintinline{latex}{foreman-installer}.

\begin{minted}{bash}
apt-get update && apt-get -y install foreman-installer
\end{minted}

Instalátor je možné spustit buď v~interaktivním módu, kdy je uživatel dotázán na nastavení na obrazovce. Druhou možností je vyplnění konfiguračního souboru \mintinline{latex}{/etc/foreman-installer/scenarios.d/foreman-answers.yaml}, ze kterého si instalátor sebere odpovědi automaticky. V~příloze práce je tento soubor vyplněn. (Vytvořen jako šablona v~Ansible).


V~interaktivním módu je vygenerováno heslo pro uživatele admin, pomocí kterého je možné se přihlásit do uživatelského webového rozhraní. Pokud použijeme Ansible z~přílohy práce, je třeba toto heslo v~konfiguračním souboru \mintinline{latex}{defaults/defaults.yaml} upravit a to bude aplikováno.


\subsubsection{Disková rozdělení}

V~teoretické části práce byly popsány výhody a nevýhody formátování oddílů disku pomocí MBR či GPT. V~instalovaných serverech touto infrastrukturou se bude používat MBR pro malé disky (do 1TB) a to z~důvodu kompatibility základních desek. Při větších kapacitách již využijeme GPT rozdělení z~důvodů popsaných v~teoretické části práce.

Práce obsahuje šablony pro operační systémy Debian a CentOS (a operační systémy z~nich odvozené) a to pro:
\begin{table}[]
\centering
\caption{Podpora diskových rozdělení}
\label{gptnmbr}
\begin{tabular}{lll}
        & Kickstart  & Preseed    \\
NO RAID & GPT \& MBR & GPT \& MBR \\
RAID1   & GPT \& MBR & GPT \& MBR \\
RAID0   & GPT \& MBR & GPT \& MBR
\end{tabular}
\end{table}


\subsection{Zálohování}

V~případě této práce je v~konfiguraci užita relační databáze PostgreSQL. V~databázi jsou uloženy  informace o~serverech, statistiky a metadata Foremanu. Díky tomu, že krátký čas odstávky vadit nebude, je možné si dovolit udělat zálohu databáze metodou dump. Obecně dump je soubor, který obsahuje strukturu tabulek a dále data obsažená v~těchto tabulkách. Dle rady v~manuálu \cite{foreman-backup} dále zálohujeme adresáře

\begin{itemize}
\item \mintinline{latex}{/etc/foreman} obsahující všechny konfigurační soubory Master Foremanu
\item \mintinline{latex}{/var/lib/puppet/ssl} -- obsahující certifikáty
\end{itemize}

Celé zálohování blíže popisuje shellový skript uložený v~\\\mintinline{latex}{/usr/local/bin/foreman-backup.sh}:
\begin{minted}{text}
#!/usr/bin/env bash

# Backup The Foreman, following the advice at
# http://theforeman.org/manuals/1.7/index.html#5.5.1Backup

set -e # If any command fails, stop this script.
ME=$(basename $0)
main () {

  DATE=$(date '+%Y%m%d.%H%M')
  BACKUPDIR=/data/backups/backup-$DATE
  mkdir $BACKUPDIR
  chgrp postgres $BACKUPDIR
  chmod g+w $BACKUPDIR

  cd $BACKUPDIR

  # Backup postgres database
  su - postgres -c "pg_dump -Fc foreman > $BACKUPDIR/foreman.dump"

  # Backup config ifles

  tar --selinux -czf $BACKUPDIR/etc_foreman_dir.tar.gz
      /etc/foreman
  tar --selinux -czf $BACKUPDIR/var_lib_puppet_dir.tar.gz
      /var/lib/puppet/ssl
  tar --selinux -czf $BACKUPDIR/tftpboot-dhcp.tar.gz
      /var/lib/tftpboot /etc/dhcp/ /var/lib/dhcpd/

  ls -lh *tar.gz foreman.dump

}

main 2>&1 | /usr/bin/logger -t $ME
\end{minted}

Tento skript je spouštěn každý den ve 2 hodiny ráno. Opakované spouštění je řešeno pro UNIX klasickým způsobem, tedy pomocí CRON démona. Řádek v~CRON tabulce pro uživatele root (otevřeme pomocí příkazu \mintinline{latex}{crontab -e}) vypadá následovně:

\begin{minted}{text}
2 0 * * *  /usr/local/bin/foreman-backup.sh
\end{minted}



Následovně ještě celý filesystem serveru zálohujeme pomocí utility backuppc \cite{backuppc}. Jediné co je pro klienta potřebné spustit, je rsync v~démon módu. Nainstalujeme tedy rsync a vytvoříme služby pro systemd.

Je potřeba vytvořit soubor  \mintinline{latex}{/etc/systemd/system/rsyncd.socket} s~obsahem:
\begin{minted}{text}
[Unit]
Description=Rsync Server Activation Socket
ConditionPathExists=/etc/rsyncd.conf

[Socket]
ListenStream=873
Accept=true

[Install]
WantedBy=sockets.target
\end{minted}

Soubor se systemd službou, nacházející se v~\\\mintinline{latex}{/etc/systemd/system/rsyncd@.service}:

\begin{minted}{text}
[Unit]
Description=Rsync Server
After=local-fs.target
ConditionPathExists=/etc/rsyncd.conf

[Service]
# Note: this requires /etc/rsyncd.conf
ExecStart=/usr/bin/rsync --daemon
StandardInput=socket
\end{minted}



Dříve, než můžeme tuto službu použít, je ještě potřeba vytvořit soubor \mintinline{latex}{/etc/rsyncd.conf}. V~nejjednodušší konfiguraci bude obsahovat:

\begin{minted}{text}
lock file = /var/run/rsync.lock
log file = /var/log/rsyncd.log
pid file = /var/run/rsyncd.pid

[foreman]
    path = /home/foreman
    uid = rsync
    gid = rsync
    read only = no
    list = yes
    auth users = rsyncclient
    secrets file = /etc/rsyncd.secrets
    hosts allow = 192.168.1.0/255.255.255.0
\end{minted}


Soubor \mintinline{latex}{/etc/rsyncd.secrets} obsahuje uživatelská jména a hesla oddělená dvojtečkou. Následovně je třeba nastavit na souboru správná přístupová práva, aby ostatní uživatelé nebyli schopni soubor číst, či nijak upravovat.
\begin{minted}{text}
# chmod 600 /etc/rsyncd.secrets
\end{minted}


\subsection{Zabezpečení serveru}

Jedním z~prvků zabezpečení je nastavení firewallu na serveru. Protože používáme operační systém Debian, jako firewall máme k~dispozici iptables. Na serveru zakážeme všechny porty mimo:

\begin{table}[h]
\centering
\caption{Otevřené porty na hlavním uzlu}
\label{my-label}
\begin{tabular}{@{}lllll@{}}
\toprule

Port &	Protokol &	Potřebné pro  \\ \midrule
53 &	TCP \& UDP &	DNS Server \\
67, 68 &	UDP &	DHCP  Server \\
69 &	UDP	&  TFTP Server \\
 443 &	TCP HTTP\/S &  Foreman UI a provisioning templates \\
3000 &	TCP	HTTP &  Foreman UI a provisioning templates \\
3306 &	TCP &	v~případě oddělené PostgreSQL databáze \\
5910 - 5930 &	TCP &	Server VNC Consoles \\
5432 &	TCP &	v~případě oddělené PostgreSQL databáze \\
8140 &	TCP &	Puppet Master \\
8443 &	TCP &	Smart Proxy, otevřená pouze pro Foreman \\ \bottomrule
\end{tabular}
\end{table}

\label{lets-encrypt}
Dalším krokem je vynucení SSL při připojení do webového rozhraní Foremanu. V~posledních letech byl představen projekt Let's Encrypt pod záštitou neziskové instituce Internet Security Research Group (ISRG). Let's Encrypt otevřená certifikační autorita (CA), nabízející digitální certrifikáty potřebné k~HTTPS (SSL/TLS) zdarma.

V~našem případě, kdy pro frontend využíváme webový server Apache (httpd) je vytvoření a obnova certifikátu velmi jednoduchá. Před vygenerováním certifikátu je nutné nastavit DNS A~záznam tak, aby mířil na IP adresu přiřazenou našemu serveru. Tento A~záznam musí být stejný jako hostname nastavený na serveru.  Postup instalace certifikátu a jeho obnovení je následovný:

Nainstalujeme balíček s~názvem certbot:

\begin{minted}{text}
# apt-get install python-certbot-apache
\end{minted}

Aktuálně už můžeme certbot klient použít. Vygenerování SSL certifikátu pro Apache je poměrně přímočaré. Klient automaticky získá a nainstaluje nový SSL certifikát pro validní domény v~Apache konfiguraci.

Interaktivní instalaci certifikátů vyvoláme pomocí příkazu:

\begin{minted}{text}
# certbot --apache
\end{minted}


Certbot projde získanou Apache konfiguraci a najde domény, které by měli být v~žádosti o~certifikát obsaženy. Interaktivně je možné jakoukoliv doménu odebrat. Utilita se dotáže na email potřebný při ztrátě klíče a dá nám na výběr ze dvou možností:

\begin{itemize}
\item povolený http a https přístup, pokud je nějaká potřeba pro nezašifrovanou komunikaci,
\item nebo přesměrování z~http na https, které my využijeme.
\end{itemize}

Když instalace proběhne, vytvořené certifikáty spolu se soukromým klíčem jsou k~dispozici v~adresáři \mintinline{latex}{/etc/letsencrypt/live}.

Zda je certifikát validní je možné zjistit například pomocí následujícího odkazu:

\begin{minted}{text}
https://www.ssllabs.com/ssltest/analyze.html?d=example.com\&latest
\end{minted}

Certifikáty vydávané autoritou Let's Encrypt jsou validní na 90 dní, proto je potřeba je po uplynulé době obnovovat. Doporučená doba k~tomu je 60 dní. Aplikace ertbot obsahuje parametr renew, pomocí kterého certifikát obnovíme.

Praktickým způsobem, jak na linuxovém stroji vykonávat nějakou činnost opakovaně, je cron démon. Protože obnovovací příkaz zjišťuje pouze datum expirace a obnoví certifikát pouze pokud do expirace zbývá méně, než 30 dní, není problém tuto činnost spustit například jednou týdně.

Úpravu CRON tabulky uživatele root provedeme následovně:

\begin{minted}{text}
# crontab -e
\end{minted}

Příkaz otevřel soubor \mintinline{latex}{/var/spool/cron/crontabs} pomocí našeho editoru. Do souboru přidáme řádek a uložíme:

\begin{minted}{bash}
30 2 * * 1 /usr/bin/certbot renew >> /var/log/le-renew.log
\end{minted}

Tato část automaticky spustí aplikaci \mintinline{latex}{/usr/bin/certbot renew} každé pondělí ve dvě ráno, v~době, kdy by server neměl být tak těžce zatížen. Informace o~průběhu se uloží do logu v~adresáři \mintinline{latex}{/var/log}.
