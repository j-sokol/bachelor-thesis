
\section{Collectd plugin}

Cílem našeho rozšíření je mít v~grafickém rozhraní systému Foreman jednoduché grafy, které zobrazují informace o~nainstalovaných serverech. Mezi tyto informace patří například vytížení procesoru, využití paměti, nebo přenos dat na síťovém rozhraní. Takových systémů na sběr dat existují stovky, proto v~rozsahu bakalářské práce nemá smysl vyvíjet další. Na pomoc jsem si vzal démon na sběr statistických dat, collectd. Nabízí mnoho doplňků, díky kterým můžeme o~serverech sbírat skoro jaékoliv informace nás napadnou.

\subsubsection{bootstrap na nově instalovaných serverech}

Na serverech, které jsou přes foreman instalovány se po prvotní provizi operačního systému nainstaluje collectd a poté nakonfiguruje v~klient režimu. V~příloze je skript, který collectd nastaví následovně:

\begin{minted}{bash}
	LoadPlugin contextswitch
	LoadPlugin cpu
	LoadPlugin df
	LoadPlugin disk
	LoadPlugin entropy
	LoadPlugin interface
	LoadPlugin load
	LoadPlugin memory
	LoadPlugin processes
	LoadPlugin uptime
	LoadPlugin users
	LoadPlugin vmem
	LoadPlugin tcpconns

	<Plugin df>
	  FSType "rootfs"
	  IgnoreSelected true
	  ReportInodes true
	</Plugin>
	LoadPlugin network
	  <Plugin network>
	  ReportStats true
	    <Server "ctd-server-ip" >
	      SecurityLevel "Encrypt"
	      Username "username"
	      Password password
	    </Server>
	</Plugin>
\end{minted}

V~této chvíli jsou již metriky (ty, co jsou vybrané mezi načtenými pluginy) odesílány na server s~IP adresou server-ip.

% TODO: skript, co ziska z API passw a pak si ho ulozi.


\subsubsection{server s~uloženými daty}

Všechna statistická data jsou uložena na dalším odděleném serveru. Tento počítač ma dle předchozí kapitoly IP adresu "ctd-server-ip". Aplikace collectd zde běží v~server režimu a naslouchá na portu 0.0.0.0:25826 na UDP. Tomu napovídá níže přiložená konfigurace v~collectd.conf:

\begin{minted}{bash}
Hostname FIXME-HOSTNAME
FQDNLookup false
Interval 30
ReadThreads 1
LoadPlugin syslog
<Plugin syslog>
        LogLevel info
</Plugin>
LoadPlugin rrdtool
LoadPlugin network

## Server config
<Plugin network>
  Listen "0.0.0.0" "25826"
  ReportStats true
	SecurityLevel "Sign"
  AuthFile "/etc/collectd/auth_file"
</Plugin>

<Plugin rrdtool>
        DataDir "/var/lib/collectd/rrd"
</Plugin>
\end{minted}

Jak můžeme vidět, uživatelská jména a hesla jsou ukládána do souboru /etc/collectd/auth\_file. Konfigurační soubor má formát

\begin{minted}{bash}
username1: pass1
username2: pass2
\end{minted}

% TODO: skript, který přidá řádek pro každý username ve foremanovi + vygeneruje heslo

% TODO: skript rest api, které pro danou IP vrátí uname + ip

\subsubsection{Collectd Graph Panel}

Pro poskytnutí grafů na serveru s~uloženými daty využijeme vynikajícího projektu Collectd Graph Panel. Tento grafický frontend je vytvořený v~jazyce PHP a distribuovaný pod licencí GPL 3. Pro instalaci potřebujeme nainstalovaný \mintinline{latex}{rrdtool}, který je možné operačním systému Debian získat následovně:
\begin{minted}{text}
# apt-get install rrdtool
\end{minted}





Dále webový server s~podporou PHP verze alespoň 5.0, použijeme tedy nám známý Apache server (httpd) s~pro nás postačujícím \mintinline{latex}{mod_php}.

\begin{minted}{text}
# apt-get install apache2 libapache2-mod-php

# a2enmod mod_php
\end{minted}


Instalace Collectd Graph Panelu je velmi jednoduchá. Vezmeme oficiální git repozitář, který naklonujeme do \mintinline{latex}{DocumentRootu} webového serveru.

\begin{minted}{text}
$ git clone https://github.com/pommi/CGP
\end{minted}

V~této podobě již CGP zobrazuje grafy. Pro naše potřeby je ale nutné ještě nakonfigurovat zabezpečené připojení na webovém serveru a omezit přístup do aplikace pouze z~určitých IP adres. Před nastavením zabezpečeného připojení potřebujeme povolit ssl modul v~Apache web serveru:
\begin{minted}{text}
# a2enmod ssl
\end{minted}

Získání certifikátu od certifikační autority Let's Encrypt je popsáno na straně \pageref{lets-encrypt}.


% Dalším krokem je vytvoření self-signed certifikátu. Nástrojem pro to bude OpenSSL, konkrétně nástroj v příkazové řádce pro vytváření certifikátů, klíčů, žádostí, atp:
% \begin{minted}{bash}

% openssl req -x509 -nodes -days 365 -newkey rsa:2048 -keyout /etc/apache2/ssl/apache.key -out /etc/apache2/ssl/apache.crt
% \end{minted}

% Zde popíši, co jednotlivé přepínače dělají.


% \begin{itemize}

% \item req je "podpříkaz" pro management X.509 certifikačních žádostí. V kryptografii je X.509 standard pro systémy založené na PKI (veřejném klíči).
% \item -x509 tento přepínač nám říká, že chceme vytvořit self-signed certifikát místo vygenerování certifikační žádosti.

% \item -nodes nám speciikuje, že nechceme zapezpečit náš klíč pomocí slovního hesla. Pokud bychom tak udělali, Apache by nebyl schopný se sám spustit a museli bychom zadat heslo vždy, když webserver nastartuje.

% \item  -days 365 nám říká, že certifikát bude platný pro příští rok.
% \item -newkey rsa:2048 nám ve stejnou chvíli vytvoří certifikační žádost a nový soukromý klíč. Parametr "rsa:2048" řekne OpenSSL, aby vygeneroval RSA klíč s délkou 2048 bitů.
% \item -keyout je název souboru s privátním klíčem
% \item -out je jmého souboru certifikátu.
% \end{itemize}


% Dalším krokem je získání certifikátu. Pro to využijeme certifikační autority Lets Encrypt. Lets Encrypt je schopen vydávat certifikáty podepsané autoritou IdentTrust, díky čemuž jsou přijímané všemi hlavními prohlížeči.


% získání certifikátu

% Let’s Encrypt validuje
% https://www.upcloud.com/support/install-lets-encrypt-apache/






\subsection{Vývoj pluginu}


\subsubsection{Název}
Dle doporučení v~dokumentaci Foremanu má název začínat předponou "foreman\_", pro lepší identifikaci a asociaci pluginu s~projektem. Dále pokud je název víceslovný, jednotlivá slova oddělujeme podtržítkem. Protože jsou pluginy publikovány jako gemy na stránce rubygems.org, je také potřeba zjistit, zna námi zvolené jméno je stále volné (tomu pomůže i zmíněný prefix).

Námi zvolené jméno je tedy "foreman\_colletctd\_graphs".

\subsubsection{Příprava prostředí}

Na GitHubu projektu je již fungující ukázkový plugin, který je možné využít pro jakékoliv naše potřeby. Obsahuje mnoho typů chování, které je nám libo využít - mezi ně patří přidání nových modelů, přepisování pohledů, přídávání uživatelských práv, položek do menu, atp. Soubor README v~ukázce obsahuje seznam aktuálního chování.

Se základními znalostmi programu git si projekt naklonujeme na lokální stroj, kde budeme projekt vyvíjet:
\begin{minted}{text}

$ git clone https://github.com/theforeman/foreman_plugin_template
 		foreman_colletctd_graphs
\end{minted}

Tímto stáhneme aktuální verzi z~master větve do adresáře foreman\_colletctd\_graphs. Repozitář obsahuje skript na změnu jména na námi zvolené, to provedeme následovně:
\begin{minted}{text}

$ ./rename.rb foreman_colletctd_graphs
\end{minted}




\subsubsection{Instalace pluginu}

Instalace pluginu ve vývojovém prostředí je jednodužším řešením, protože kód náčítá za běhu a nené potřeba instalovat plugin jako gem. Vytvořením vývojové instance Foremanu jsme se zabývali v~kapitole ZZ.

Plugin je možné rovnou spustit (a prozkoumat jeho chování) úpravou souboru \mintinline{latex}{Gemfile.local.rb}. Také je možné vytvořit soubor \\\mintinline{latex}{foreman_colletctd_graphs.rb} v~adresáři \mintinline{latex}{bundler.d} a vložit do něj tento řádek:

\begin{minted}{bash}
# Gemfile.local.rb
gem 'foreman_colletctd_graphs',
 		:path => 'path_to/foreman_colletctd_graphs'
\end{minted}

Potom instalujeme "preface" bundle příkazem:
\begin{minted}{text}
$ bundle install
\end{minted}

A~znovu restartujeme Foreman. Nový plugin můžeme vidět v~záložce plugin tab na About Page.

\subsubsection{Vývoj}

Prvně upravíme soubor \mintinline{latex}{foreman_colletctd_graphs.gemspec} a do něj vložíme meta informace, jako jméno pluginu, autora, web stránku projektu, verzi. Dále do souboru \mintinline{latex}{lib/foreman_colletctd_graphs/engine.rb} vložíme následující:

\begin{minted}{bash}

# lib/foreman_colletctd_graphs/engine.rb
initializer 'foreman_colletctd_graphs.register_plugin',
 		:before => :finisher_hook do |app|
  Foreman::Plugin.register :foreman_colletctd_graphs do
    # the code of our plugin
  end
end
\end{minted}



% \subsubsection{Struktura pluginu}
%
% The project is a Rails::Engine [27] as described in the instructions for Foreman plugins
% in the official documentation [54]. The most adequate choice was an engine with a
% semi-isolated namespace [46], since I wanted to avoid namespace pollution resulting in
% name conflicts but I needed an access to the classes in Foreman and Katello. The plugin
% has a standard Rails project structure with minor modifications mirroring the conventions
% in Foreman and Katello projects.
\subsubsection{Deface}

Pro úpravu HTML stránek, do kterých budeme vkládat grafy, použijeme knihovnu Deface. Umožňuje nám upravit HTML pohledy bez zásahu do spodního Ruby pohledu. Použití knihovny je možné dvěma způsoby, já zde popíši jen jeden z~nich. Pro více informací je dostupná dokumentace zde \cite{deface}.


\subsubsection{Content Security Policy}

\mintinline{latex}{Content-Security-Policy} je HTTP hlavička vytvořená s~hlavním cílem snížení XSS rizik deklarováním, jaký obsah je povolen k~načtení. V~současnosti hlavičku podporují prohlížeče Google Chrome (od verze 25+), Firefox (31+), Safari (7+), i Microsoft Edge.

Ve frameworku Foreman je tato hlavička vynucena -- pomocí gemu Secure Headers. Pokud grafy vygenerované z~rrd souborů chceme načítat z~jiného serveru, do konfigurace pluginu je třeba přidat adresu serveru s~uloženými daty. Konfigurace se provádí v~souboru \\\mintinline{latex}{/usr/share/foreman/config/initializers/secure_headers.rb} a to následovně:



% \begin{figure}[h]\centering
% \includegraphics[width=1\textwidth]{files/truthiestmeme.jpg}
% 	\caption{Zásada programování}\label{fig:float}
% \end{figure}




%
%
% \subsubsection{Vytvoření Gemu}
%
% http://bundler.io/v1.12/guides/creating_gem.html
%
%
% \subsubsection{Balíček pro OS Debian}
%
% https://softwareengineering.stackexchange.com/questions/195633/good-approaches-for-packaging-php-web-applications-for-debian
%



\subsection{Nasazení pluginu}

V~produkčním prostředí je nasazení pluginu velmi jednoduché. Náš vytvořený .deb balíček pomocí rsync/scp technologie přesuneme na server. V~adresáři obsahující balíček napíšeme:
\begin{minted}{bash}

$ rsync -avh foreman_colletctd_graphs.deb $UNAME@$SERVERNAME:
\end{minted}

a následovně nainstalujeme nástrojem standartně obsaženým v~Debianu:
\begin{minted}{text}

$ dpkg -i foreman_colletctd_graphs.deb
\end{minted}

Dalším krokem je restartovat Foreman server, pokud máme systemd jako náš init systém, tento krok vykonáme následovně:
\begin{minted}{text}
# systemctl restart foreman.service
\end{minted}

V~této chvíli plugin máme aktivní a funkční.

\subsection{Závěr}

Ačkoliv výsledný plugin neobsahuje hodně kódu, veškerá jeho funkcionalita dle zadání je naplněna. Na obrázku v~příloze je možné vidět aktuální podobu.

Po naprogramování práce jsem objevil již vytvořený plugin foreman-colly \cite{foreman-colly}, též generující grafy sbírané démonem collectd, ale fungující na odlišném způsobu. Doplněk od Lukáše Zapletala aktivně naslouchá paketům s~metrikami, můj plugin pouze zobrazuje již vygenerované grafy. Výhledově by bylo možné tyto dvě funcionality spojit do jednoho většího doplňku.
